% -*- coding: utf-8 -*-
% \documentclass[journal]{IEEEtran}
\documentclass[titlepage, twocolumn, a4paper, 10pt]{article}
\usepackage[english]{babel}
\usepackage[utf8]{inputenc}
\usepackage{verbatim}
\usepackage{fancyhdr}
\usepackage{graphicx}
\usepackage{parskip}
\usepackage{url}

\usepackage[pdfborder={0 0 0 0}]{hyperref}

% Column spacing
\setlength{\columnsep}{10mm}

% Include pdf with multiple pages ex \includepdf[pages=-, nup=2x2]{filename.pdf}
\usepackage[final]{pdfpages}

% Place figures where they should be use [H]
\usepackage{float}

% Float for text
\floatstyle{ruled}
\newfloat{code}{H}{lop}
\floatname{code}{CodeSnippet}

% vars
\def\title{GCom}
\def\preTitle{Deliverable 2}
\def\kurs{Distributed systems, HT-09}


\def\namn{Anton Johansson}
\def\mail{dit06ajn@cs.umu.se}

\def\namnTva{Jonny Strömberg}
\def\mailTva{dit06jsg@cs.umu.se}


\def\pathtocode{\url{~/dit06ajn/edu/dist/GCom}}

\def\handledareEtt{Lars Larsson, larsson+ds@cs.umu.se}
\def\handledareTva{Daniel Henriksson, danielh+ds@cs.umu.se}

\def\inst{Computer Science}
\def\dokumentTyp{Report}

\begin{document}
\begin{titlepage}
  \thispagestyle{empty}
  \begin{small}
    \begin{tabular}{@{}p{\textwidth}@{}}
      UMEÅ UNIVERSITY \hfill \today \\
      Department of \inst \\
      \dokumentTyp \\
    \end{tabular}
  \end{small}
  \vspace{10mm}
  \begin{center}
    \LARGE{\preTitle} \\
    \huge{\textbf{\kurs}} \\
    \vspace{10mm}
    \LARGE{\title} \\
    \vspace{15mm}
    \begin{large}
      \namn, \mail \\
      \namnTva, \mailTva\\
      \texttt{\pathtocode}
    \end{large}
    \vfill
    \large{\textbf{Supervisors}}\\
    \mbox{\large{\handledareEtt}}\\
    \mbox{\large{\handledareTva}}
  \end{center}
\end{titlepage}

\newpage
\mbox{}
\vspace{70mm}
\begin{center}
  % Dedication goes here
\end{center}
\thispagestyle{empty}
\newpage

\pagestyle{fancy}
\rhead{\today}
\lhead{\footnotesize{\namn, \mail\\\namnTva, \mailTva}}
\chead{}
\lfoot{}
\cfoot{}
\rfoot{}

\cleardoublepage
\newpage
\onecolumn
\tableofcontents
\twocolumn
\cleardoublepage

\fancyfoot[LE,RO]{\thepage}
\pagenumbering{arabic}

\section{Introduction}\label{sec:intro}
% Beskriv med egna ord vad uppgiften gick ut på. Är det någonting som
% varit oklart och ni gjort egna tolkningar så beskriv dessa.
This report explains a solution for implementing a distributed
group communications middleware.

A distributed system is composed of separated processes that
coordinate activities by passing messages and a middleware is a
software layers that enables rapid development of other software by
supplying simple method-calls that hides the underlying implementation
details off the middleware.

The middleware described in this report is called \textit{GCom} and
provides an API\footnote{Application programming interface} for group
communication with different message sending/delivery rules. Two
communication methods are implemented: \textit{Reliable multicast},
\textit{Basic multicast}, described in greater detail in section
\ref{sec:communications-module}.

Four message-ordering types are implemented: \textit{Non-ordered},
\textit{First in first out}, \textit{Casual}, \textit{Total} and
\textit{Casual-Total}, described in greater detail in section
\ref{sec:message-ordering-module}.

The original specification of this practical assignment can be found at:\\
\begin{footnotesize}
  \url{http://www.cs.umu.se/kurser/5DV020/HT09/assignment.html}
  \footnote{Fetched \today} % DONE check
\end{footnotesize}

\section{Problem analysis}\label{sec:problem-analysis}
% As this project emphasizes analysis and investigation of a loosely
% specified problem, include any assumptions you made during the
% analysis phase in your report. Also discuss problems encountered and
% alternative solutions considered in the analysis. The report should
% also discuss to what extent the requirement list is fulfilled, as
% well as to which extent you could adhere to the the project plan.

\section{Usage}\label{sec:usage}
% Förklara var programmet och källkoden ligger samt hur man kompilerar,
% startar och använder det. Förklara även översiktligt vad som händer
% när man använder de olika kommandona. Det räcker alltså inte att
% skriva "man skriver 'ant' för att kompilera", utan det måste även ingå
% en liten förklaring om vad som egentligen händer när man kör ant och
% varför det fungerar. Använd Internet eller litteratur för att själva
% ta reda på den information ni tycker känns relevant, dels för
% rapportens skull och dels för er egen. Kom ihåg att skriva tydliga
% (vetenskapliga) referenser!
All files needed to use this middleware are located at:\\
\texttt{\pathtocode}

This catalog contains the following sub directories:
\begin{itemize}
\item The directory \verb!src! contains the source code.
\item The directory \verb!src/main/resources/! contains configuration
  files for standard behaviour of the compiled system, see section
  \ref{sec:configuration}
\item The directory \verb!src! contains the source code.
\item The directory \verb!bin! will, after a successful compilation,
  contain all the compiled sources as well as configuration files used
  by this middleware.
\item The directory \verb!lib! contains all requires third-party libraries
  needed by \textit{GCom}, se section \ref{sec:required-libraries}.
\item The directory \verb!doc! contains the Javadoc API for \textit{GCom}.
\end{itemize}

\subsection{Configuration}\label{sec:configuration}
The compiled system uses two configuration-files to define its
standard behaviour, these files are located in the directory
\textit{src/main/resources/}.

\subsubsection{application.properties}\label{sec:application.properties}
The file \textit{application.properties} defines the standard
multicast and ordering types to use when communication with a group.
Notice though that these settings are only used for the creator of a
group that did not exist from before. When connection to an existing
group, the settings from that group will suppress the settings in
\textit{application.properties}. CodeSnippet \ref{code:app-prop}
shows the content of an example configuration that uses

\begin{code}
  \begin{footnotesize}
\begin{verbatim}
# Used by GNS
gcom.gns.port=1078

# FIFO, TOTAL_ORDER, NO_ORDERING,
# CASUAL_ORDERING, CASUALTOTAL_ORDERING
gcom.ordering=FIFO

# BASIC_MULTICAST, RELIABLE_MULTICAST
gcom.multicast=RELIABLE_MULTICAST
\end{verbatim}
  \end{footnotesize}
  \caption{applications.properties}\label{code:app-prop}
\end{code}


\section{Compilation}\label{sec:compilation}
The following commands will require the software tool \textit{Apache
  Ant}\footnote{http://ant.apache.org/}. More details about what
happens using \textit{ant} in this project is found in the file
\textit{build.xml}\footnote{http://ant.apache.org/manual/using.html}.

To compile \textit{GCom} issue the following command:\\
\begin{footnotesize}
  \verb!salt:./GCom> ant!
\end{footnotesize}\\
This will create a directory \verb!bin! if it does not already exists
and compile/move source-code and configuration files to that
directory.

The root-directory for class-files when using \textit{GCom} is
compiled to \textit{bin/main/java}, while the root-directory for
test-code is compile to \textit{bin/test/java}.

To create \textit{jar}-file of the compiled sources issue the
following command:\\
\begin{footnotesize}
  \verb!salt:./GCom> ant jar!
\end{footnotesize}\\
This will create \textit{GCom.jar} which can be used when developing
in third party software or directly as a \textit{GNS}-server (see
section \ref{sec:group-naming-system}) by
running:\\
\begin{footnotesize}
  \verb!salt:./GCom> java -jar GCom.jar!
\end{footnotesize}

\subsection{Required libraries}\label{sec:required-libraries}
Internally \textit{GCom} uses some third party software located in the
\textit{lib} directory and described in the following sections.

\subsection{SLF4J and Logback}\label{sec:logback}

\subsection{JUnit}\label{sec:junit}
For testing the individual parts of \textit{GCom}, tests are written
using the \textit{JUnit testing
  framework}\footnote{http://www.junit.org/}.

\section{System description}\label{sec:system}
% Beskriv översiktligt hur programmet är uppbyggt och hur det löser
% problemet.

% The GCom middleware consists of three (logical) modules, the group
% management module, the communication module and the message ordering
% module. These are, respectively, responsible for handling group
% membership issues, communication message exchange semantics and
% message (re)ordering issues. All of these modules need to function
% properly in order for your system to be able to ensure correct
% message delivery semantics.

% \begin{figure}[!t]
%   \centering
%   \includegraphics[width=2.5in]{images/Stack.pdf}
%    %   \centerline{\subfloat[Case I]\includegraphics[width=2.5in]{images/Stack.pdf}}
%   \caption{GCom stack}
%   \label{fig:images/Stack}
% \end{figure}

\begin{figure*}[!t]
  \centerline{\includegraphics[width=110mm]{images/Stack.pdf}}
  \caption{GCom stack}
  \label{fig:images/Stack}
\end{figure*}



\subsection{Group management module}\label{sec:group-management-module}

\subsubsection{Group Naming System}\label{sec:group-naming-system}
% To resolve group names: When a process sends a message to the group,
% the group management module resolves the group name into a list of
% group members.

\subsubsection{Group leaders}\label{sec:group-leaders}
% To provide an interface for group management: The group management
% module provides operations to create and remove groups, as well as
% add and remove members from a group.

\subsubsection{Error handling}\label{sec:error-handling}
% To detect errors: The module monitors a group and indicates when a
% member of the group crashes (or for some other reason become
% unreachable).

\subsubsection{Group changes}\label{sec:group-changes}
% To notify changes in group membership: The module notifies all group
% members about changes in group composition.


\subsection{Communications module}\label{sec:communications-module}
\subsubsection{Basic multicast}\label{sec:basic-multicast}
\subsubsection{Reliable multicast}\label{sec:reliable-multicast}


\subsection{Message ordering module}\label{sec:message-ordering-module}
\subsubsection{Non-ordered}\label{sec:-non-ordered}
\subsubsection{FIFO}\label{sec:fifo}
\subsubsection{Causal}\label{sec:causal}
\subsubsection{Total}\label{sec:total}
\subsubsection{Causal-Total}\label{sec:causal-total}
% (messages are first sorted according to causal ordering, then according to total ordering)


\subsection{Debugger}\label{sec:debugger}
% That messages are delivered according to the specified ordering

% How messages are propagated in the network: Show both the path a
% message takes and how many times a certain process has received a
% certain message.

% The content of hold-back queues and other buffers: Present all
% messages waiting to be sent or delivered as well as values of vector
% clocks and other counters.

% Current system performance: As a measure of the system performance,
% count the number of messages (including control messages) required
% to perform an operation (send one message with certain ordering and
% certain multicast).


\section{Limitations}\label{sec:limitations}
% Vilka problem och begränsningar har din lösning av uppgiften? Hur
% skulle de kunna rättas till?

% \section{Reflektioner}\label{Reflektioner}
% % Reflektioner - Var det något som var speciellt krångligt? Vilka
% % problem uppstod och hur löste ni dem? Vilka verktyg använde ni? Hur
% % upplevde ni de verktygen? + Allmänna synpunkter. Om ni har upplevt
% % problem på grund av olika miljöer (i termer av operativsystem och
% % liknande) så kan det även vara intressant att nämna det, samt motivera
% % ert val av miljö.

\section{Tests}\label{sec:tests}
% Noggranna testkörningar där man ser att programmet fungerar som det
% ska.

% During your demo, you will need to convince the teachers that your
% implementation works. Bring a test protocol, i.e., a series of tests
% that clearly demonstrates that your GCom fulfills the requirements
% and a test tool which can be used to apply it. The test protocol
% should include, e.g., tests of all message orders and multicast
% types. Bring a copy of the test protocol on paper, see page 491 in
% [DS] for suggested notation. Your test protocol must clearly state
% your names, user names, and which level you intend to demonstrate.

% The fact that a system cannot be formally proven to work does not
% make it impossible to implement - consider for example the Internet.
% Read pages 498 and 508 in [DS].

%%%%%%%%%%%%%%%% END APPENDIX AND STUFF %%%%%%%%%%%%%%%%
\bibliographystyle{alpha}
\bibliography{books.bib}

\newpage
\appendix
\pagenumbering{roman}
\section{Appendix}\label{sec:kallkod}
% Källkoden ska finnas tillgänglig i er hemkatalog
% ~/edu/apjava/lab1/. Bifoga även utskriven källkod.

\end{document}
